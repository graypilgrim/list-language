\documentclass{article}
\usepackage{polski}
\usepackage[utf8]{inputenc}
\usepackage{secdot}
\sectiondot{subsection}
\usepackage{scrextend}
\usepackage{booktabs}
\usepackage{array}
\usepackage{ltablex}
\usepackage{amsmath}
\usepackage{amsfonts}
\usepackage{listings}
\usepackage{parskip}
\addtokomafont{labelinglabel}{\sffamily}
\usepackage{color}
\definecolor{light-gray}{gray}{0.95}
\newcommand{\keyword}[1]{\colorbox{light-gray}{\texttt{#1}}}
\newcommand{\code}[1]{\texttt{#1}}
\usepackage{textcomp}

\title{\vspace{7cm}\LARGE Techniki kompilacji\\semestr 17L\\Język do operacji na listach}
\author{\LargeŁukasz Wlazły\\nr albumu: 269365}
\date{}

\begin{document}
\maketitle
\pagenumbering{gobble}
\newpage
\pagenumbering{arabic}

\section{Założenia projektowe}
\begin{enumerate}
\item Język będzie posiadał statyczne typowanie.
\item Język będzie obsługiwał wszystkie typy liczbowe dostępne w C++11 oraz wbudowany typ listy, który będzie przechowywał proste typy liczbowe.
\item Typy proste będą przekazywane do funkcji przez wartość, listy natomiast będą przekazywane przez referencję.
\item Nie będzie możliwe zagnieżdżanie funkcji, deklaracja może odbywać się tylko w zasięgu globalnym.
\item Program rozpoczyna się od funkcji \code{main} o typie \code{bool}.
\end{enumerate}

\section{Atomy leksykalne}
\keyword{list}, \keyword{if}, \keyword{else}, \keyword{while}, \keyword{for}, \keyword{in}, \keyword{return}\\
\keyword{true}, \keyword{false}, \\
\keyword{(}, \keyword{)}, \keyword{[}, \keyword{]}, \keyword{\{}, \keyword{\}}, \\
\keyword{=}, \keyword{+}, \keyword{-}, \keyword{*}, \keyword{/} \\
\keyword{==}, \keyword{!=}, \keyword{<=}, \keyword{>=}, \\
\keyword{\&\&}, \keyword{||}, \\
\keyword{;}

\section{Gramatyka}
\code{program} \textrightarrow \code{funsDefs} \\
\code{funsDefs} \textrightarrow \code{funDef | funDef funsDefs} \\
\code{funDef} \textrightarrow \code{funDecl\{ exprs \} | funDecl\{ exprs return val\}} \\
\code{funDecl} \textrightarrow \code{varDecl( args )} \\
\code{args} \textrightarrow \code{varDecl | varDecl, args} \\
\code{exprs} \textrightarrow \code{expr | expr\keyword{;} exprs} \\
\code{expr} \textrightarrow \code{condExpr | loopExpr | callExpr | indexExpr | varDef | varDecl | listExpr} \\

\code{condExpr} \textrightarrow \code{ifExpr | ifExpr elseExpr} \\
\code{ifExpr} \textrightarrow \code{\keyword{if (} logicStmt \keyword{)} exprs } \\
\code{elseExpr} \textrightarrow \code{\keyword{else} exprs} \\
\code{logicStmt} \textrightarrow \code{val logicOp val | logicStmt logicTie logicStmt} \\

\code{loopExpr} \textrightarrow \code{whileExpr | forExpr} \\
\code{whileExpr} \textrightarrow \code{\keyword{while (} logicStmt \keyword{)} exprs} \\
\code{forExpr} \textrightarrow \code{\keyword{for (} varDef \keyword{;} logicStmt \keyword{;} assignExpr \keyword{)} exprs}

\code{listExpr} \textrightarrow \code{\keyword{[} basicListIter \keyword{]} | \keyword{[} basicListIter condExpr \keyword{]} } \\
\code{basicListIter} \textrightarrow \code{ expr \keyword{for} identifier \keyword{in} identifier} \\

\code{varDef} \textrightarrow \code{varDecl \keyword{=} val} \\
\code{varDecl} \textrightarrow \code{type identifier} \\
\code{val} \textrightarrow \code{lVal | rVal} \\
\code{lVal} \textrightarrow \code{identifier | indexExpr} \\
\code{rVal} \textrightarrow \code{identifier | callExpr | indexExpr | identifier mathOp val | number | boolVal} \\
\code{indexExpr} \textrightarrow \code{identifier\keyword{[}number\keyword{]}} \\

\code{identifier} \textrightarrow \code{letter*} \\
\code{logicOp} \textrightarrow \code{\keyword{==} | \keyword{!=} | \keyword{<=} | \keyword{>=}} \\
\code{logicTie} \textrightarrow \code{\keyword{\&\&} | \keyword{||}} \\
\code{mathOp} \textrightarrow \code{\keyword{=} | \keyword{+} | \keyword{-} | \keyword{*} | \keyword{/}} \\
\code{letter} \textrightarrow \code{\keyword{A} \ldots \keyword{z}} \\
\code{number} \textrightarrow \code{digit* | digit*.digit*} \\
\code{digit} \textrightarrow \code{\keyword{0} \ldots \keyword{9}} \\


\end{document}
