\documentclass{article}
\usepackage{polski}
\usepackage[utf8]{inputenc}
\usepackage[top=3cm, bottom=3cm, left=3cm, right=3cm]{geometry}
\usepackage{secdot}
\sectiondot{subsection}
\usepackage{scrextend}
\usepackage{booktabs}
\usepackage{array}
\usepackage{ltablex}
\usepackage{amsmath}
\usepackage{amsfonts}
\usepackage{listings}
\usepackage{parskip}
\addtokomafont{labelinglabel}{\sffamily}
\usepackage{color}
\definecolor{light-gray}{gray}{0.95}
\newcommand{\keyword}[1]{\colorbox{light-gray}{\texttt{#1}}}
\newcommand{\code}[1]{\texttt{#1}}
\usepackage{textcomp}

\title{\vspace{7cm}\LARGE Techniki kompilacji\\semestr 17L\\Język do operacji na listach}
\author{\LargeŁukasz Wlazły\\nr albumu: 269365}
\date{}

\begin{document}
\maketitle
\pagenumbering{gobble}
\newpage
\pagenumbering{arabic}

\section{Założenia projektowe}
\begin{enumerate}
\item Język będzie posiadał statyczne typowanie.
\item Język będzie obsługiwał typy liczbowe (int oraz float), typ logiczny (bool) oraz brak typu zwracanego dla funkcji (void).
\item Typy proste będą przekazywane do funkcji przez wartość, listy natomiast będą przekazywane przez referencję.
\item Nie będzie możliwe zagnieżdżanie deklaracji funkcji, deklaracja może odbywać się tylko w zasięgu globalnym.
\item Program rozpoczyna się od funkcji \code{main} o typie \keyword{bool}.
\item Komentarzem będą wszystkie znaki w danej linii, które zaczynają się od znaku \keyword{@}
\end{enumerate}

\section{Słowa kluczowe}
\keyword{list}, \keyword{if}, \keyword{else}, \keyword{while}, \keyword{for}, \keyword{in}, \keyword{return}\\
\keyword{true}, \keyword{false}, \\
\keyword{(}, \keyword{)}, \keyword{[}, \keyword{]}, \keyword{\{}, \keyword{\}}, \\
\keyword{=}, \keyword{+}, \keyword{-}, \keyword{*}, \keyword{/} \\
\keyword{==}, \keyword{!=}, \keyword{<=}, \keyword{>=}, \\
\keyword{\&\&}, \keyword{||}, \\
\keyword{;}
\newpage
\section{Gramatyka}
\code{program} \textrightarrow \code{funsDefs | funsDecls} \\
\code{funsDefs} \textrightarrow \code{funDef | funDef funsDefs} \\
\code{funDef} \textrightarrow \code{type identifier\{ stmts \} | type identifier\{ stmts retStmt \}} \\
\code{funsDecls} \textrightarrow \code{funDecl | funsDecls} \\
\code{funDecl} \textrightarrow \code{type identifier( args ) \keyword{;}} \\
\code{args} \textrightarrow \code{varDecl | varDecl, args} \\
\code{stmts} \textrightarrow \code{stmt | stmt\keyword{;} | \keyword{\{} stmts \keyword{\}} | stmt\keyword{;} stmts} \\
\code{stmt} \textrightarrow \code{condStmt | loopStmt | callStmt | indexStmt | varDef | varDecl | listStmt | retStmt | assignStmt} \\

\code{condStmt} \textrightarrow \code{ifStmt | ifStmt elseStmt} \\
\code{ifStmt} \textrightarrow \code{\keyword{if (} expr \keyword{)} stmts } \\
\code{elseStmt} \textrightarrow \code{\keyword{else} stmts} \\
\code{retStmt} \textrightarrow \code{\keyword{return} val} \\
\code{callStmt} \textrightarrow \code{identifier\keyword{(}vals\keyword{)}} \\

\code{loopStmt} \textrightarrow \code{whileStmt | forStmt} \\
\code{whileStmt} \textrightarrow \code{\keyword{while (} expr \keyword{)} stmts} \\
\code{forStmt} \textrightarrow \code{\keyword{for (} \{varDef | $\epsilon$\} \keyword{;} \{expr | $\epsilon$\} \keyword{;} \{expr | $\epsilon$\} \keyword{;)} stmts}

\code{listStmt} \textrightarrow \code{\keyword{[]} | \keyword{[} basicListIter \keyword{]} | \keyword{[} basicListIter listIfStmt \keyword{]} } \\
\code{listIfStmt} \textrightarrow \code{\keyword{if (} expr \keyword{)}} \\
\code{basicListIter} \textrightarrow \code{ stmt \keyword{for} identifier \keyword{in} identifier} \\

\code{varDef} \textrightarrow \code{varDecl \keyword{=} val} \\
\code{varDecl} \textrightarrow \code{type identifier} \\
\code{listDecl} \textrightarrow \code{\keyword{list(}type\keyword{)} identifier} \\
\code{vals} \textrightarrow \code{val | val, vals } \\
\code{val} \textrightarrow \code{lVal | rVal} \\
\code{lVal} \textrightarrow \code{identifier | indexStmt} \\
\code{rVal} \textrightarrow \code{identifier | callStmt | indexStmt | listStmt | expr | number | boolVal} \\
\code{indexStmt} \textrightarrow \code{identifier\keyword{[}identifier\keyword{]} | identifier\keyword{[}intNumber\keyword{]}} \\
\code{assignStmt} \textrightarrow \code{identifier \keyword{=} rVal} \\

\code{expr} \textrightarrow \code{andExpr | andExpr \keyword{||} andExpr} \\
\code{andExpr} \textrightarrow \code{compExpr | compExpr \keyword{\&\&} compExpr} \\
\code{compExpr} \textrightarrow \code{sumExpr | sumExpr compOperator sumExpr} \\
\code{sumExpr} \textrightarrow \code{mulExpr | mulExpr sumOperator mulExpr} \\
\code{mulExpr} \textrightarrow \code{val | val mulOperator val} \\

\code{identifier} \textrightarrow \code{letter*} \\
\code{type} \textrightarrow \code{\keyword{int} | \keyword{float} | \keyword{bool} | \keyword{void}} \\
\code{sumOperator} \textrightarrow \code{\keyword{+} | \keyword{-}} \\
\code{mulOperator} \textrightarrow \code{\keyword{*} | \keyword{/}} \\
\code{compOperator} \textrightarrow \code{\keyword{==} | \keyword{!=} | \keyword{<=} | \keyword{>=}} \\
\code{letter} \textrightarrow \code{\keyword{A} \ldots \keyword{z}} \\
\code{number} \textrightarrow \code{intNumber | floatNumber} \\
\code{intNumber} \textrightarrow \code{digit*} \\
\code{floatNumber} \textrightarrow \code{digit*.digit*} \\
\code{digit} \textrightarrow \code{\keyword{0} \ldots \keyword{9}} \\

\newpage
\section{Przykłady}
\subsection{Proste użycie wywołania funkcji}
\begin{lstlisting}[tabsize=2]
int add(int x, int y) {
	return x + y;
}

int main() {
	int a = 6;
	int b = 7;
	int wynik = add(a, b);

	return false;
}
\end{lstlisting}

\subsection{Funkcja rekurencyjna}
\begin{lstlisting}[tabsize=2]
int factorial(int x) {
	if (x == 1)
		return 1;
	return x*factorial(x-1);
}
\end{lstlisting}

\subsection{Operacje na listach}
\begin{lstlisting}[tabsize=2]
int factorial(int x) {
	if (x <= 1)
		return 1;
	return x*factorial(x-1);
}

int main() {
	list(int) a = [5];
	for (int i = 0; i < 5; i=i+1)
		a[i] = i;

	list(int) f = [factorial(i) for i in a];

	return false;
}

\end{lstlisting}

\end{document}
